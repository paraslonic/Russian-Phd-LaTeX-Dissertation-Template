
{\actuality} В настоящее время накапливается все больше данных о том, что геном прокариот представляет из себя сложно организованную нуклеотидную последовательность. Помимо кодирующих и регуляторных областей, в нем имеется ряд элементов, необходимых для взаимодействия ДНК с молекулярными комплексами, осуществляющими процессы транскрипции и репликации. Пространственная укладка генетического материала также не случайна и выполняет ряд регуляторных функций.Это делает геном отличным от простого хранилища последовательностей генов, расположенных в случайном порядке, и позволяет говорить об его архитектуре --- закономерностях, которые необходимы для успешного функционирования живой клетки.

К настоящему времени известен ряд элементов геномной организации. Гены, расположенные рядом с сайтом начала репликации, представлены в большем количестве копий в быстро делящихся клетках, что делает выгодным расположение генов, продукты которых необходимы клетке в больших количествах, поблизости от сайта начала репликации. Пространственная укладка ДНК может сближать гены, расположенные в разных областях линейной последовательности, что может быть полезно для генов, кодирующих регулятор и его мишени. Экспериментально было установлено, что действие глобальных регуляторов, таких как гистоноподобный белок H-NS, зависит от местоположения генов-мишеней. Склонность к транскрипции (уровень экспрессии генов, не зависящий от их последовательности) значительно меняется в зависимости от положения гена в хромосоме. Взаимодействие РНК-полимераз, возникающее за счет изменения уровня суперскрученности ДНК, может играть роль в регуляции транскрипции соседних генов.

Геномные перестройки и горизонтальный перенос генов могут приводить к изменению оптимального расположения генов и других элементов генома, что может приводить к снижению жизнеспособности организма. Известно, что изменения в геномах преимущественно локализуются в отдельных местах --- "горячих"\ точках. Возможно, эти участки свободны от "архитектурных"\ ограничений, и таким образом более толерантны к изменениям. Возможно, эти участки имеют некоторые признаки, способствующие более высокой частоте происходящих изменений. Нельзя исключать возможности, что "горячие точки"\ возникли в результате генетического дрейфа а их расположение случайно и не обладает функциональным значением. Какие из этих вариантов, и в какой степени, реализуются в действительности, к настоящему моменту, неизвестно: локализация "тихих"\ консервативных участков и "горячих"\ высокоизменчивых локусов, насколько нам известно, не имеет объяснений. Для проведения исследований в данной области необходим инструмент, позволяющий находить и анализировать области генома с повышенной и пониженной изменчивостью. Разработка и применение подобного инструментария и стала основной темой данной работы. 

{\aim} данной работы является разработка программного конвейера для выявления высокоизменчивых областей прокариотических геномов и применение его для анализа изменчивости в локусах генома \textit{E. coli}, ассоциированных с болезнью Крона.


Для~достижения поставленной цели необходимо было решить следующие {\tasks}:
\begin{enumerate}[beginpenalty=10000] % https://tex.stackexchange.com/a/476052/104425
  \item Разработать алгоритм графового представления расположения генов в геномах и оценки уровня изменчивости генома.
  \item Сравнить профили изменчивости геномов, принадлежащих различным родам, видам и подвидовым структурам прокариот.
  \item Оценить вклад различных факторов геномной организации в уровень изменчивости генома. 
  \item Разработать алгоритм визуализации подграфов соответствующих отдельным локусам генома.
  \item Разработать алгоритм поиска, провести поиск оперонов кишечной палочки, наличие которых ассоциировано с наличием у человека-носителя болезни Крона.  
\end{enumerate}


{\novelty}
Предложенный в нашей работе подход, насколько нам известно, был первым предложенным и реализованным методом для количественной оценки изменчивости генома. 

Насколько нам известно, мы впервые провели сравнительный анализ расположения областей повышенной изменчивости. Мы обнаружили, что некоторые локусы генома остаются высокоизменчивыми у всех представителей вида, в то время, как ряд других локусов являются таковыми лишь у некоторых видов и филогрупп. 

{\influence} 

Изменчивость генома --- важный фактор в возникновении патогенных штаммов бактерий и приобретения устойчивости к антибиотикам. Знание закономерностей подобных изменений важно для разработки оптимальных методов контроля над появлением штаммов бактерий, угрожающих жизни и здоровью людей. Возможно, полученные знания о закономерностях изменчивости и консервативности различных областей генома окажутся полезными при создании новых последовательностей геномов в области синтетической биологии. 

%{\methods} 

{\defpositions}
\begin{enumerate}[beginpenalty=10000] % https://tex.stackexchange.com/a/476052/104425
	\item Графовое представление геномов является эффективным средством для визуального анализа изменений в генном составе и поиска областей генома с повышенной изменчивостью.

	\item Геномы из различных филогрупп и филогенетически близких видов обладают как консервативными (расположенными в схожих местах генома), так и вариативными (присутствующими лишь у некоторых представителей) областями повышенной изменчивости.

 	\item Наибольшее количество областей повышенной изменчивости в рассмотренных геномах находится в областях локализации профагов.

	\item Опероны, значимо чаще встречающиеся в изолятах \textit{E. coli}, изолированных от пациентов с болезнью Крона по сравнению с изолятами полученными от здоровых людей, расположены как в низко-изменчивых областях генома (опероны захвата сорбозы, захвата гемина, утилизации глиоксилата), так и в высокоизменчивых областях (оперон утилизации пропандиола, синтеза и экспорта капсульных полисахаридов).

\end{enumerate}


{\reliability} предложенного метода обосновывается результатами компьютерного моделирования. Результаты находятся в соответствии с результатами, полученными другими авторами. Основные результаты работы докладывались~на конференциях: "Итоговая научно-практическая конференция ФГБУ ФНКЦ ФХМ ФМБА России" (18-19 декабря 2019 года, Москва), “ПОСТГЕНОМ 2018”\  (29 октября - 2 ноября, Казань),"Биотехнология: состояние и перспективы развития"\ (Москва, 20–22 февраля 2017 года), "Высокопроизводительное секвенирование в геномике"\ (Новосибирск, 18–23 июня 2017 года), "4th World Congress on Targeting Microbiota", Париж,  17-19 октября 2016).


{\contribution} 
Автором были предложены подходы графового представления набора генов в геномах и оценки геномной вариабельности на основе выбора подграфа. Написан код на языках R, perl и Snakemake для графового представления набора геномов и автоматизации анализа геномных последовательностей (исправлении ошибок в гомополимерных областях, поиска контаминаций в наборе прочтений, построения ортогрупп, филогенетического анализа). Проведена сборка последовательностей геномов изолятов \textit{E. coli} полученных от пациентов с болезнью Крона и проведено сравнение их с комменсальными штаммами. Проведен анализ расположения областей повышенной изменчивости у различных видов и внутривидовых структурах прокариот.

\ifnumequal{\value{bibliosel}}{0}
{%%% Встроенная реализация с загрузкой файла через движок bibtex8. (При желании, внутри можно использовать обычные ссылки, наподобие `\cite{vakbib1,vakbib2}`).
    {\publications} Основные результаты по теме диссертации изложены
    в~14~печатных изданиях,
    6 из которых изданы в журналах, рекомендованных ВАК,
    8 "--- в тезисах докладов.
}%
{%%% Реализация пакетом biblatex через движок biber
    \begin{refsection}[bl-author, bl-registered]
        % Это refsection=1.
        % Процитированные здесь работы:
        %  * подсчитываются, для автоматического составления фразы "Основные результаты ..."
        %  * попадают в авторскую библиографию, при usefootcite==0 и стиле `\insertbiblioauthor` или `\insertbiblioauthorgrouped`
        %  * нумеруются там в зависимости от порядка команд `\printbibliography` в этом разделе.
        %  * при использовании `\insertbiblioauthorgrouped`, порядок команд `\printbibliography` в нём должен быть тем же (см. biblio/biblatex.tex)
        %
        % Невидимый библиографический список для подсчёта количества публикаций:
        \printbibliography[heading=nobibheading, section=1, env=countauthorvak,          keyword=biblioauthorvak]%
        \printbibliography[heading=nobibheading, section=1, env=countauthorwos,          keyword=biblioauthorwos]%
        \printbibliography[heading=nobibheading, section=1, env=countauthorscopus,       keyword=biblioauthorscopus]%
        \printbibliography[heading=nobibheading, section=1, env=countauthorconf,         keyword=biblioauthorconf]%
        \printbibliography[heading=nobibheading, section=1, env=countauthorother,        keyword=biblioauthorother]%
        \printbibliography[heading=nobibheading, section=1, env=countregistered,         keyword=biblioregistered]%
        \printbibliography[heading=nobibheading, section=1, env=countauthorpatent,       keyword=biblioauthorpatent]%
        \printbibliography[heading=nobibheading, section=1, env=countauthorprogram,      keyword=biblioauthorprogram]%
        \printbibliography[heading=nobibheading, section=1, env=countauthor,             keyword=biblioauthor]%
        \printbibliography[heading=nobibheading, section=1, env=countauthorvakscopuswos, filter=vakscopuswos]%
        \printbibliography[heading=nobibheading, section=1, env=countauthorscopuswos,    filter=scopuswos]%
        %
        \nocite{*}%

        %
        {\publications} Основные результаты по теме диссертации изложены в~\arabic{citeauthor}~печатных изданиях,
        \arabic{citeauthorvak} из которых изданы в журналах, рекомендованных ВАК\sloppy%
        \ifnum \value{citeauthorscopuswos}>0%
            , \arabic{citeauthorscopuswos} "--- в~периодических научных журналах, индексируемых Web of~Science и Scopus\sloppy%
        \fi%
        \ifnum \value{citeauthorconf}>0%
            , \arabic{citeauthorconf} "--- в~тезисах докладов.
        \else%
            .
        \fi%
        \ifnum \value{citeregistered}=1%
            \ifnum \value{citeauthorpatent}=1%
                Зарегистрирован \arabic{citeauthorpatent} патент.
            \fi%
            \ifnum \value{citeauthorprogram}=1%
                Зарегистрирована \arabic{citeauthorprogram} программа для ЭВМ.
            \fi%
        \fi%
        \ifnum \value{citeregistered}>1%
            Зарегистрированы\ %
            \ifnum \value{citeauthorpatent}>0%
            \formbytotal{citeauthorpatent}{патент}{}{а}{}\sloppy%
            \ifnum \value{citeauthorprogram}=0 . \else \ и~\fi%
            \fi%
            \ifnum \value{citeauthorprogram}>0%
            \formbytotal{citeauthorprogram}{программ}{а}{ы}{} для ЭВМ.
            \fi%
        \fi%
        % К публикациям, в которых излагаются основные научные результаты диссертации на соискание учёной
        % степени, в рецензируемых изданиях приравниваются патенты на изобретения, патенты (свидетельства) на
        % полезную модель, патенты на промышленный образец, патенты на селекционные достижения, свидетельства
        % на программу для электронных вычислительных машин, базу данных, топологию интегральных микросхем,
        % зарегистрированные в установленном порядке.(в ред. Постановления Правительства РФ от 21.04.2016 N 335)
    \end{refsection}%
    \begin{refsection}[bl-author, bl-registered]
        % Это refsection=2.
        % Процитированные здесь работы:
        %  * попадают в авторскую библиографию, при usefootcite==0 и стиле `\insertbiblioauthorimportant`.
        %  * ни на что не влияют в противном случае


    \end{refsection}%
 
    
    %
        % Всё, что вне этих двух refsection, это refsection=0,
        %  * для диссертации - это нормальные ссылки, попадающие в обычную библиографию
        %  * для автореферата:
        %     * при usefootcite==0, ссылка корректно сработает только для источника из `external.bib`. Для своих работ --- напечатает "[0]" (и даже Warning не вылезет).
        %     * при usefootcite==1, ссылка сработает нормально. В авторской библиографии будут только процитированные в refsection=0 работы.
}

