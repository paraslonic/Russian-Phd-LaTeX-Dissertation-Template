\graphicspath{{Dissertation/images/}}

\chapter{Обсуждение. } \label{chaptDiscussion}


В настоящее время, для некоторых видов прокариот доступны сотни и даже тысячи геномов. Постоянно пополняющуюся коллекцию геномных последовательностей можно использовать для получения информации о вариабельности генома, его архитектуре, устройства различных оперонов, геномных островов и иных мобильных элементов.

Инструменты визуализации синтении (например, Mauve, BRIG, genePlotR) часто используются для сравнительных исследований геномов прокариот и вирусов. Они позволяют визуально определить наличие больших и малых перестроек генома, но применимы для сравнительно небольших количеств сравниваемых геномов (10-20), а при большем размере выборки визуальный анализ становится затруднителен. В настоящее время существует недостаток методов визуализации сравнения большого (превышающих десятки) количества геномов и численной оценки изменчивости геномов. Подобного рода инструменты необходимы для исследования структуры генома, отдельных геномных элементов и изучения факторов, влияющих на возникновении либо исчезновение высокоизменчивых областей генома и областей с пониженной изменчивостью.

Для эффективного анализа и визуализаций больших наборов геномов мы предложили подход на основе графов, в котором гены представлены в виде узлов, узлы связываются между собой ребром, если соответствующие гены расположены последовательно в одном из геномов. Такое представление позволяет создавать компактные визуализации изменчивости определенного локуса в десятках и сотнях геномов. Мы также предложили использовать подсчет путей на графе для оценки изменчивости геномных последовательностей. 

Графы имеют давнюю историю применения в анализе геномов. Они давно применяются для сборки геномов из множества коротких прочтений. При этом узлами графа служат либо сами прочтения, либо отдельные подстроки фиксированной длины (k-меры). Поиск оптимального обхода на графе дает оптимальное выравнивание прочтений (либо k-меров), что позволяет воссоздать последовательность генома. Применяются также методы картирования прочтений на набор референсных геномов, представленных в виде графа. Такой подход позволяет учесть уже известные варианты геномов для более полного картирования прочтений. 

Графы применялись также для анализа изменчивости генома. В нескольких работах группы Е. Кунина использовалось графовое представление расположения кластеров ортологичных групп (COG) во множестве геномов. Значительное распространение получило представление перестроек генома в виде графов разрыва (brakepoints graph), удобное для реконструкции предковых состояний генома, но не для задач визуализации. 

В нашей работе, мы использовали представление набора геномов в виде графа для двух задач: численной оценки уровня изменчивости в отдельных локусах генома и визуализации подграфов, соответствующих отдельным областям генома. Визуализация подграфов позволяет дать ответы на ряд вопросов о контексте генов интереса. Например, находится ли ген или гены интереса в одинаковом окружении во всех рассматриваемых геномах? Какие альтернативные генные контексты существуют и в каких геномах они представлены? Какие части набора генов (например, оперона или генного острова) являются консервативными, а какие вариабельными? Какие геномы содержат определенную комбинацию генов? 

Мы провели поиск оперонов, которые чаще встречаются у кишечных палочек, выделенных из образцов фекалий и кишечных смывов пациентов с болезнью Крона - тяжелого воспалительного заболевания кишечника. Функция большинства найденных оперонов ясна, они позволяют микробу захватывать железо, утилизировать пропандиол (продукт переработки слизистого слоя), менять антигенные свойства, тем самым убегая от иммунного ответа. Интересно, что анализ графов, представляющих контекст этих генов, показал очень разные картины. Два оперона: утилизации пропандиола и производства капсулы находятся в высокоизменчивых -- "горячих"\ -- областях генома. Опероны утилизации гимина, утилизации глиоксилата, захвата сорбозы напротив находятся в "тихих"\ областях. Роль генетических факторов, находящихся в областях генома с разным уровнем изменчивости, в формировании генотипа и фенотипа бактерий --- предмет дальнейших исследований. В случае оперона синтеза и экспорта капсульных полисахаридов, можно предположить, что нахождение данного оперона в "горячей"\ области генома может способствовать более высокой изменчивости состава оперона (у него есть высоко вариативная часть, отвечающая за синтез капсулы), что в свою очередь выгодно для эффективного избегания иммунного ответа организма-хозяина.

При помощи графового представления геномов мы реализовали метод количественной оценки локальной изменчивости, основанный на поиске уникальных путей в подграфе. Под изменчивостью в данном случае мы понимаем изменение состава либо взаимного расположения генов в геноме. Под локальностью --- то, что изменения затрагивают небольшую область генома, не превышающую размер выбранного окна анализа (выбирается пользователем, обычно, составляет около 20-40 генов). Насколько нам известно, разработанный нами вычислительный конвейер (Genome Complexity Browser, GCB) является первым доступным инструментом, который позволяет количественно определять изменчивость генома на основе заданного пользователем набора геномов. GCB предоставляет способ оценки профиля изменчивости вдоль репликонов, что позволяет находить "горячие точки"\ генома, в которых уровень изменчивости значительно выше, чем в остальной части генома, и его "тихие"\ области. 

Нам представляется перспективным направление исследований, включающее поиск изменений в интенсивности и местоположении "горячих"\ и "тихих"\ областей. Такой анализ можно проводить при помощи сравнения профилей изменчивости на разных уровнях филогенетического сходства: между геномами из различных внутривидовых структур (например, клад филогенетического дерева или экотипов) или между геномами близких видов. Результатом подобных исследований должно стать знание факторов, влияющих на уровень изменчивости генома в отдельных его областях. GCB позволяет оценивать изменчивость в определенном локусе генома (на основании задаваемых пользователем близкородственных геномов). Размер локуса может задаваться параметром окна, указываемом при запуске программы. Анализ геномов при помощи скользящего окна позволяет получить профиль уровня изменчивости вдоль генома. Мы предусмотрели метод автоматического определения областей повышенной изменчивости на основе критерия Тьюки (критерий учитывает медианное значение и межквартильный размах).

Для проверки применимости предложенного метода мы провели вычислительные эксперименты по моделированию эволюции бактериальных геномов. Мы допустили возможность вставки, удаления, перемещения генов, а также геномные инверсии. Вероятность этих событий была не равномерна вдоль модельной хромосомы, но менялась в соответствии с задаваемым профилем. Таким образом, в основе моделирования лежало предположение, что в геноме существуют области повышенной и пониженной изменчивости, положение которых устойчиво во времени. Затем мы оценивали профиль изменчивости на основе полученных геномов и предложенного нами подхода. Мы наблюдали высокий уровень сходства исходных и вычисленных профилей. Вероятность геномных инверсий в наших экспериментах была на два порядка ниже, чем остальные события, что согласуется с наблюдаемой частотой у многих бактериальных видов. Этот параметр крайне важен для применимости метода: для эффективной оценки локальной изменчивости, глобальные перестройки генома не должны нарушать понятие локальности. 

В применимости нашего подхода нас также убеждает анализ областей генома, содержащих известные "горячие точки"\ изменчивости: интегроны и геномные острова. Данные области обладают высоким уровнем изменчивости при анализе предложенным методом, резко выделяясь на фоне прилегающей части генома. 

Профиль изменчивости генома \textit{E. coli} обладает ярко выраженной контрастностью. На нем видны протяженные участки с низкой изменчивостью и области, в которых изменчивость очень высока. Большинство областей с высокой изменчивостью совпадает с локализацией профагов, либо описанных ранее островов патогенности. Существуют также "горячие"\ точки, включая и наиболее изменчивые области, внутри которых нам не удалось обнаружить следов мобильных элементов (такое же наблюдение описано в исследовании \cite{oliveira2017chromosomal}). Протяженность низко-изменчивых областей оказалась выше (наиболее протяженная область составляет порядка миллиона пар нуклеотидов), высокоизменчивые области как правило короче (наиболее длинные области достигают длины в 200-300 т.п.н.).

Мы сравнили профили изменчивости геномов \textit{E. coli} из пяти основных филогрупп: A, B1, B2, D и E. Для каждой филогруппы был выбран один референсный геном и сто ближайших к нему геномов из базы RefSeq, для каждого набора из 101 геномы мы провели анализ изменчивости независимо. Референсные геномы обладали достаточно высоким уровнем сохранности структуры генома (без крупных перестроек), что позволило сравнивать расположение областей повышенной и пониженной изменчивости. Сравнение показало, что множество областей повышенной изменчивости расположено в близких по контексту областях генома. Значительная часть этих областей соответствует профаговым областям, так что их консервативное расположение можно объяснить сайт-специфичным характером их встраивания и изменчивостью самих фаговых геномов. Особенно заметна роль профагов в вариабельности геномов из филогруппы E (референсный геном: \textit{O157:H7 Sakai}). Интересно, что области интеграции разных фагов значительно различаются по уровню своей изменчивости. Ряд областей повышенной изменчивости геномов, не содержащих профагов, также показал консервативность расположения.

Мы провели сравнение профилей изменчивости для представителей различных филогрупп у трех других видов бактерий. Для \textit{P. aeruginosa} и \textit{N. gonorrhoeae} мы наблюдали, что области повышенной изменчивости расположены в сходном геномном контексте. Иная картина наблюдалась в случае {P. fluorescens}, для которой сходство профилей было незначительным, а сами профили отличались равномерностью (отсутствием ярко выраженных областей с повышенной и пониженной изменчивостью). На наш взгляд, это можно объяснить высокой частотой геномных перестроек, в частности - инверсий, зафиксированных в геномах данного вида. Как уже отмечалось выше, высокая частота крупных геномных перестроек делает наш подход неприменимым к анализу локальной изменчивости геномов. 

Мы также провели сравнение профилей изменчивости для геномов из филогенетически близких видов и наблюдали ряд "горячих точек"\, расположенных в близких местах генома и ряд областей повышенной изменчивости, специфичных для отдельных видов. 

Предложенный в нашей работе подход к оценке геномной изменчивости не универсален. Например, он не подходит для обнаружения крупных геномных перестроек (больше, чем параметр окна, который обычно составляет несколько десятков генов) или изменений в некодирующих частях генома. Точность и применимость нашего подхода зависит от точности поиска ортологичных генов --- сложной вычислительной задачи, часто не имеющей однозначного решения. В нашем исследовании мы использовали инструмент OrthoFinder, который использует алгоритм кластеризации графов MCL для значений попарного сходства белков, нормированных на длину выравнивания. Мы считаем этот инструмент оптимальным на настоящий момент с точки зрения эффективности и точности. Недостатком данного подхода является частое попадание паралогичных генов в одну ортогруппу, что затрудняет дальнейший анализ. Мы наблюдали, что в среднем 0,5\% всех ортогрупп на геном содержат по крайней мере один паралогичный ген; среди всех ортогрупп, предполагаемых для вида, доля ортогрупп с паралогами составляет почти 16\%. 

Мы реализовали два подхода к анализу ортогрупп, содержащих паралогичные гены. Подход, реализованный как метод по умолчанию в GCB, заключается в игнорировании таких ортогрупп. Другой подход заключается в искусственной ортологизации паралогов (каждый паралогичный ген с уникальным левым и правым контекстом генов добавляется в граф с уникальным суффиксом). Исходя из нашего опыта, оптимальной стратегией является использование режима по умолчанию для первоначального анализа с последующей проверкой всех выводов в режиме ортологизации. 

В GCB предусмотрено использование двух алгоритмов автоматической компоновки (layout) узлов графов: Dagre и Graphviz. Все же, для обеспечения наиболее ясной компоновки часто требуется ручные выравнивание. Новые алгоритмы автоматической компоновки, возможно, позволят снизить необходимость "ручного"\ выравнивания.

Несмотря на вышеупомянутые недостатки, мы считаем, что предложенный метод анализа изменчивости геномов уже информативен и применим в достаточной степени, для проведения дальнейшего анализа факторов, влияющих на локализацию областей повышенной и пониженной изменчивости у прокариот.

Горячие точки изменчивости генома были описаны для нескольких видов бактерий. В работе \cite{oliveira2017chromosomal} авторы проанализировали области генома, в которых часто наблюдается последствия горизонтального переноса генов. Они провели анализ для 80 видов бактерий и пришли к выводу, что во многих "горячих точках"\ отсутствуют мобильные генетические элементы, и предположили, что гомологичная рекомбинация в первую очередь ответственна за изменчивость этих локусов. Аналогичный вывод о ведущей роли в горизонтальном переносе генов процесса гомологичной рекомбинации без участия мобильных элементов был сделан в работе \cite{ely2020recombination}. Факторы, определяющие расположение высокоизменчивых точек, влияющие на их появление и устранение, остаются открытыми вопросами. Являются ли эти области повышенной изменчивости теми местами, где действительно чаще происходят изменения? Альтернативным является предположение, что изменения происходят значительно равномернее вдоль генома, но организмы, в которых изменения произошли в неподходящих местах не выживают и мы их не наблюдаем.

Одной из исходных гипотез данного исследования являлась связь между уровнем изменчивости и пространственной укладкой хромосомы. Нам удалось найти лишь слабую связь между профилем межхромосомных контактов и уровнем изменчивости. Попытка сопоставить уровень изменчивости с реконструированной пространственной укладкой хромосомы кишечной палочки (\cite{hacker2017features}) также не дала значимого результата (что также может быть связано с несовершенством моделей пространственной укладки). Весьма вероятно, что на уровень изменчивости разных локусов генома влияет не один, но значительное количество различных факторов. Ряд этих факторов могут быть связаны с архитектурой генома. С меньшей вероятностью будут выживать организмы, в которых изменения в геномах привели к нарушениям в архитектуре и нарушению клеточных процессов с участием хромосомы и иных элементов генома. В таком случае, дальнейшие исследования изменчивости геномов могут пролить свет на еще не известные элементы геномной архитектуры и способствовать более эффективному конструированию генов в рамках программы синтетической биологии. С другой стороны, можно предположить, что распределение областей высокой изменчивости носит случайный характер, например в следствии случайного распределения различных сайтов интеграции мобильных элементов, часть которых может быть не известна. Выяснение вклада различных факторов в расположение более и менее изменчивых областей --- дело будущих исследований. 

\chapter{Выводы}

\begin{enumerate}
    \item Графовое представление геномов позволяет эффективно проводить поиск областей генома с повышенной изменчивостью.
  
      \item Геномы представителей различных филогрупп и филогенетически близких видов прокариот имеют консервативно расположенные области повышенной изменчивости (расположенные в местах генома с одинаковым генным контекстом).
  
       \item Уровень геномной изменчивости ассоциирован с плотностью хромосомных контактов (коэффициент корреляции составил -0.36) и плотностью расположения сайтов Chi (коэффициент корреляции составил -0.25).
  
      \item Следующие опероны значимо чаще (p-value < 0.0001) встречаются в изолятах \textit{E. coli} от пациентов с болезнью Крона: захват сорбозы, захват гемина, утилизации глиоксилата, утилизации пропандиола, синтеза и экспорта капсульных полисахаридов.
  
      \item Оперон утилизации пропандиола и оперон синтеза и экспорта капсульных полисахаридов расположены в высокоизменчивых областях, а опероны захвата сорбозы, захвата гемина и утилизации глиоксилата --- в консервативных участках генома \textit{E. coli}.
    
    \end{enumerate}